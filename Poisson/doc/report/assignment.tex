\documentclass[11pt]{article}
\usepackage{xeCJK}
\usepackage{geometry}
\usepackage{bm}
\usepackage{amsmath}
\usepackage{graphicx}
\usepackage{hyperref}
\usepackage{subfigure}
\usepackage{datetime}
\usepackage{fontspec}
\usepackage{fancyhdr}
\usepackage{indentfirst}
\usepackage{titlesec}
\usepackage[round]{natbib}
\usepackage{float}
\usepackage{booktabs}

% global style
\geometry{b5paper,left=0.6in,right=0.6in,top=1in,bottom=0.8in}
\linespread{1.3}
\setlength{\parindent}{2em}

\renewcommand{\today}{\small \number\year 年 \number\month 月 \number\day 日}
\renewcommand{\abstractname}{\normalsize 摘\ \ 要}
\renewcommand{\contentsname}{目录}
\renewcommand{\refname}{参考文献}
\renewcommand{\sectionmark}[1]{\markright{第\,\thesection\,章\, #1}}

% title style
\titleformat*{\section}{\Large\bfseries}
\titleformat*{\subsection}{\large\bfseries}
\titleformat*{\subsubsection}{\normalsize\bfseries}

% font style
\setCJKmainfont{SimSun}[BoldFont=SimHei, ItalicFont=KaiTi]
\setmainfont{Times New Roman}[BoldFont=Arial]

% page style
\fancypagestyle{assignment}{
    \fancyhf{}
    \fancyfoot[C]{\thepage}
    \fancyhead[L]{\small 廖紫默 (SA21005043)}
    \fancyhead[C]{\small \textbf{计算流体力学}}
    \fancyhead[R]{\small 2022年春}
    \renewcommand{\headrulewidth}{0.4pt}
    \renewcommand{\footrulewidth}{0.4pt}
}
\pagestyle{assignment}

% ref style (in chinese)
\def\equationautorefname{式}%
\def\footnoteautorefname{脚注}%
\def\itemautorefname{项}%
\def\figureautorefname{图}%
\def\figurename{图}
\def\tableautorefname{表}%
\def\tablename{表}
\def\partautorefname{篇}%
\def\appendixautorefname{附录}%
\def\chapterautorefname{章}%
\def\sectionautorefname{节}%
\def\subsectionautorefname{小节}%
\def\subsubsectionautorefname{subsubsection}%
\def\paragraphautorefname{段落}%
\def\subparagraphautorefname{子段落}%
\def\FancyVerbLineautorefname{行}%
\def\theoremautorefname{定理}%

% header
\title{\LARGE\textbf{计算流体力学 大作业}\\\Large\textbf{Poisson方程求解}}
\author{\normalsize 廖紫默 (SA21005043)\\
\small 近代力学系,中国科学技术大学\\
\small \href{mailto:zimoliao@mail.ustc.edu.cn}{zimoliao@mail.ustc.edu.cn}}
\date{\today}

\begin{document}

% titlepage
\maketitle
\thispagestyle{assignment}

% part 2
\pagenumbering{arabic}
\setcounter{page}{1}
\section{问题简述}
求解Poisson方程:
\begin{equation}
	\frac{\partial^2\phi}{\partial x^2}+\frac{\partial^2\phi}{\partial y^2}=\sin{x}\cos{y},\quad 0\leq x\leq1,\ 0\leq y\leq1\label{eqn:poisson}
\end{equation}
边界条件为:
\begin{equation}
	\phi=\left\{\begin{array}{ll}
		0                          & ,x=0 \\
		y-\frac{\sin{1}\cos{y}}{2} & ,x=1 \\
		-\frac{\sin{x}}{2}         & ,y=0 \\
		x-\frac{\sin{x}\cos{1}}{2} & ,y=1 \\
	\end{array}\right.\label{eqn:bc}
\end{equation}

绘制等值线$\phi=0.05,\ 0.2,\ 0.5,\ 0.75,\ 1$。要求采用方法(迭代误差取$10^{-6}$):
\begin{enumerate}
	\item 线性方程组求解:Jacobi,G-S选一;SOR,线SOR,块SOR选一。
	\item 加速方法:CG,MG选一。
\end{enumerate}

\section{方程离散}
采用有限差分方法离散\autoref{eqn:poisson},空间二阶导项使用中心差分近似:
\begin{equation}
	\left(\frac{\partial^2\phi}{\partial x^2}\right)_i\approx\frac{\phi_{i+1}+\phi_{i-1}-2\phi_i}{\left(\Delta x\right)^2}
\end{equation}


\begin{figure}[htbp]
	\centering
	\includegraphics[width=.2\linewidth]{figure/computational_molecular.png}
	\caption{\label{fig:molecular}五点计算单元(computational molecular)\citep{ferziger_computational_2020}}
\end{figure}

\begin{figure}[htbp]
	\centering
	\includegraphics[width=.7\linewidth]{figure/algebra_equation_system.png}
	\caption{\label{fig:algebra}代数方程组与带状系数矩阵形式\citep{ferziger_computational_2020}}
\end{figure}

原Poisson方程转化为一代数方程组,记为$\bm{A}\bm{\phi}=\bm{Q}$,其中系数矩阵$\bm{A}$为一带状矩阵,由五点计算单元构成,见\autoref{fig:molecular}、\autoref{fig:algebra}。各节点($P$)上求解:
\begin{equation}
	A_W\phi_W+A_S\phi_S+A_P\phi_P+A_N\phi_N+A_E\phi_E=Q_P
\end{equation}
采用二阶中心差分的Poisson方程系数为:
\begin{equation}
	A_W=A_S=A_N=A_E=1,\quad A_P=-4
\end{equation}
此外,非齐次项由$Q_P=h^2\sin(x_P)\cos(y_P)$确定,$h=\Delta x=\Delta y$为均匀网格间距。边界条件(\autoref{eqn:bc})采用直接转移法施加到代数方程组中。

\section{代数方程组求解}
\subsection{迭代解法}
线性代数方程组迭代方法的基本思路在于将原始方程:
\begin{equation}
	\bm{A}\bm{\phi}=\bm{Q}
\end{equation}
中系数矩阵拆分$\bm{A}=\bm{M}-\bm{N}$,进而构造迭代方程\footnote{更一般的形式为:$\bm{M}\bm{\phi}^{n+1}=\bm{N}\bm{\phi}^n+\bm{B},\ \bm{PA}=\bm{M-N},\ \bm{B}=\bm{PQ}$。}:
\begin{equation}
	\bm{M}\bm{\phi}^{n+1}=\bm{N}\bm{\phi}^n+\bm{Q}
\end{equation}
当$\bm{M}^{-1}\bm{N}$的谱半径小于1时,上述迭代式可以收敛到原方程的解。构造不同的$\bm{M}$、$\bm{N}$矩阵可以得到性能相异的迭代方法。

\subsection{Jacobi方法(J)}
记$\bm{A}=\bm{L+D+U}$,分别表示下三角、对角、上三角部分,Jacobi方法直接取$\bm{M}=\bm{D}$。迭代方程如下\cite{sauer_numerical_2018}:
\begin{equation}
	\bm{\phi}^{n+1}=\bm{D}^{-1}\left(\bm{Q}-\left(\bm{L+U}\right)\bm{\phi}^n\right)
\end{equation}

对于离散Poisson方程(具有带状系数矩阵),迭代式可写为:
\begin{equation}
	\phi_P^{n+1}=-\frac{Q_P-\phi_S^n-\phi_W^n-\phi_N^n-\phi_E^n}{4}
\end{equation}

\subsection{Gauss-Seidel方法(GS)}
Gauss-Seidel方法的思路在于将迭代更新的$\phi^{n+1}$立即用于此后各点的计算中,迭代方程如下:
\begin{equation}
	\bm{\phi}^{n+1}=\bm{D}^{-1}\left(\bm{Q}-\bm{U}\bm{\phi}^n-\bm{L}\bm{\phi}^{n+1}\right)
\end{equation}

对于离散Poisson方程,迭代式可写为:
\begin{equation}
	\phi_P^{n+1}=-\frac{Q_P-\phi_S^{n+1}-\phi_W^{n+1}-\phi_N^n-\phi_E^n}{4}
\end{equation}
事实上G-S方法可以看作下文SOR方法的一个特例。

\subsection{逐次超松弛方法(SOR)}
逐次超松弛方法在G-S方法的基础上引入超松弛系数$\omega>1$,进而显著加速迭代过程,如下所示:
\begin{equation}
	\phi_P^{n+1}=\omega\phi_P^{n+1,G-S}+(1-\omega)\phi_P^n
\end{equation}
当$\omega=1$时退化为G-S方法,通常在$1.6\leq\omega\leq1.9$时有很好的加速效果,而在$\omega=2$时迭代发散。常规SOR方法(含G-S)的一大缺陷在于其每步迭代中各点的值存在依赖性,继而难以实现算法并行化。

\subsection{线SOR方法(LSOR)}
前述方法均为显式迭代方法,考虑到三对角矩阵直接求解具有极高的效率(见\autoref{ch:tdma}),线SOR方法在单步迭代中,逐行(列)构造三对角矩阵隐式求解,进而实现加速。以行SOR(记为LSOR-X)为例,即在SOR迭代式中用$\phi_E^{n+1}$替代$\phi_E^n$:
\begin{equation}
	\phi_P^{n+1}=\omega\left(-\frac{Q_P-\phi_S^{n+1}-\phi_W^{n+1}-\phi_N^n-\phi_E^{n+1}}{4}\right)+(1-\omega)\phi_P^n
\end{equation}
\begin{equation}
	-\frac{\omega}{4}\phi_W^{n+1}+\phi_P^{n+1}-\frac{\omega}{4}\phi_E^{n+1}=\omega\frac{\phi_S^{n+1}-Q_P}{4}+(1-\omega)\phi_P^n
\end{equation}
其中$\phi_S^{n+1}$为前一行已经计算得到的值,即右端项均已知,得到当前行上的代数方程组,其系数矩阵为三对角矩阵,可以采用TDMA直接求解。列SOR(LSOR-Y)同理,单个迭代步内逐列求解。

\subsection{块SOR方法(LSOR-ADI)}

\subsection{三对角矩阵直接解法(TDMA)\label{ch:tdma}}

\subsection{不完全LU分解方法(strongly implicit procedure, SIP)}

\section{加速方法}
\subsection{共轭梯度法(CG)}

\subsection{多重网格法(MG)}

\newpage
\bibliographystyle{unsrtnat}
\bibliography{ref}

\end{document}